\documentclass[letter,12pt]{article}
\usepackage[paperheight=27.94cm,paperwidth=21.59cm,bindingoffset=0in,left=3cm,right=2.0cm, top=3.5cm,bottom=2.5cm, headheight=200pt, headsep=1.0\baselineskip]{geometry}
\usepackage{ragged2e}
\usepackage{float}
\usepackage{graphicx,lastpage}
\usepackage{upgreek}
\usepackage{censor}
\usepackage[spanish,es-tabla]{babel}
\usepackage{pdfpages}
\usepackage{tabularx}
\usepackage{graphicx}
\usepackage{adjustbox}
\usepackage{xcolor}
\usepackage{colortbl}
\usepackage{rotating}
\usepackage{multirow}
\usepackage{indentfirst}
\usepackage{mdframed}
\usepackage[utf8]{inputenc}
\usepackage{listings}
\usepackage{pgfplots}
\usepackage{pgfplotstable}
\usepackage{siunitx}
\renewcommand{\tablename}{Tabla}
\usepackage{fancyhdr}
\pagestyle{fancy}
\setlength{\parindent}{1cm}

\begin{document}

\lhead{\begin{picture}(0,0) \put(0,0){\includegraphics[width=40mm]{Images/Universidad Logo.png}} \end{picture}}
\begin{figure}[t]
\includegraphics[width=8cm]{Images/Universidad Logo.png}
\centering
\end{figure}

    \title{\Huge{Estructura de Datos: Laboratorio x}}

    \author{Nombre(s) Estudiante(s)\\Profesor: Nombre Apellido\\Ayudante: Nombre Apellido\\Sección x}

    \date{(dia) de (mes) (año)}

\maketitle

\newpage
    \tableofcontents
\newpage

\section{Introduction}

En esta parte se debe señalar qué se hará en forma de resumen e información importante para dar un buen contexto a lo siguiente.

También se deben señalar las tecnologías/materiales a utilizar en el desarrollo. En este caso por ejemplo nombrar que se programará en el lenguaje Java y se trabajará con estructuras como LinkedList o similares, etc.

\newpage

\section{Desarrollo}

En el desarrollo se debe mostrar todo el procedimiento realizado para llegar al objetivo de cada actividad, explicar cada código y porqué tu codigo cumple con cada objetivo.

Lo óptimo es detallar cada ítem demostrando el código o utilizando imagenes, y lo escrito debe ser no demasiado extenso ya que esto debe ir al grano pero explicandolo de la mejor manera (se entiende según el tema que algunas explicaciones/textos serán largos).

\subsection{Subsecciones}

La utilización se subsecciones para distribuir el tema y marcar los cambios es muy recomendable, de esta manera se notan de mejor manera los distintos temas que puede tocar el desarrollo o las diferentes etapas por las que pasó el trabajo de los objetivos.

\subsection{Ejemplo de inserción de imagen}

\begin{figure}[H]
    \centering
    \includegraphics[width=1\textwidth]{Images/Universidad Logo.png}
    \caption{Ejemplo de inserción de imagen}
\end{figure}

\newpage

\section{Análisis}

En esta sección se debe análizar profundamente lo visto en el desarrollo, no en todos los informes aplica este apartado, normalmente existe en informes que realicen comparaciones o toma de muestras, etc.

Se aplicar para dar detalles de resultados o hacer comparaciones a través de gráficos o tablas, y sus explicaciones deben ser detalladas.

\subsection{Ejemplo de tabla}

\begin{table}[H]
\centering
\begin{tabular}{|l|l|l|}
\hline
Tiempos 1 & Tiempos 2 &  \\ \hline
5 minutos y 48 segundos   & 4 minutos y 34 segundos & Función 1 \\ \hline
1 minuto y 42 segundos & 1 minuto y 39 segundos & Función 2 \\ \hline
18 segundos y 708 milisegundos & 27 segundos y 084 milisegundos & Función 3 \\ \hline
\end{tabular}
\caption{Ejemplo de tabla}
\end{table}

\subsection{Ejemplo de gráfico}

\begin{figure}[H]
\centering
\begin{tikzpicture}
\begin{axis}[
    ybar,
    bar width=10,
    xlabel={Datos},
    ylabel={Tiempo (segundos)},
    symbolic x coords={'Tiempos 1', 'Tiempos 2'},
    xtick=data,
    x tick label style={rotate=45,anchor=east},
    enlarge x limits=0.5,
    ]
    
    \addplot coordinates {('Tiempos 1', 4.215) ('Tiempos 2', 40.197)};
    
\end{axis}
\end{tikzpicture}
\caption{Ejemplo de gráfico}
\label{fig:barras_milisegundos}
\end{figure}

\newpage

\section{Conclusión}

Se realiza un mini-resumen de lo realizado en las actividades, su desarrollo y los resultados (y opiniones sobre estas), si eran lo esperado o no, etc. También se señalan conocimientos nuevos gracias a la experiencia u comentarios de esta.

\end{document}